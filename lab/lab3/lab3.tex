\documentclass{article} % Defines the document class, article is commonly used
\usepackage[shortlabels]{enumitem}
\usepackage{amsmath}    % Allows for more advanced math formatting
\usepackage{amssymb}    % Provides additional mathematical symbols
\usepackage{amsthm}     % \qed
\usepackage{graphicx}   % image
\usepackage{float}      % image placement
\usepackage{hyperref}
\hypersetup{
    colorlinks=true,       % false: boxed links; true: colored links
    linkcolor=black,       % color of internal links
}
\usepackage[margin=1.5in]{geometry}
\usepackage{siunitx}

\begin{document}

\title{EEC133 Lab 3 Report}
\author{Tao Wang}
\date{\today}

\maketitle

\section*{Pre-lab}
\addcontentsline{toc}{section}{Pre-lab}
\textbf{Questions}
\begin{enumerate}[(1)]
    \item $W = 20 \si{mm}$. $L = 30 \si{mm}$.
    \item Bandwidth is $\boxed{0.098 GHz}$ and is $\boxed{4.0\%}$ of the resonant frequency. The input impedance is $46.40 - j28.53$ according to Figure 2
    \item $L_{in} = 10.6 \si{mm}$. $\text{Gap }=1 \si{mm}$. $L_{feed} = 33.1397 \si{mm}$. $\text{Line width} = 2.99231 \si{mm}$
    \item
          \begin{enumerate}[(a)]
              \item Return Loss from 2 to 3 GHz
                    \begin{figure}[H]
                        \centering
                        \includegraphics[width=0.7\textwidth]{./image/figure1.png}
                        \caption{$S_{1, 1}$ of Patch Antenna in dB}
                    \end{figure}
                    The plot shows how much of the incoming wave is reflected from the antenna. A smaller $S_{11}$ is desirable because it implies little input wave is reflected.
              \item The bandwidth is in the range of 2.388 to 2.486 GHz. The antenna will work as a bluetooth antenna because the bandwidth is greater than the bluetooth bandwidth, 2.402-2.48 GHz.
              \item Input impedance from 2 to 3 GHz
                    \begin{figure}[H]
                        \centering
                        \includegraphics[width=0.7\textwidth]{./image/figure2.png}
                        \caption{re(Z(1, 1)) and im(Z(1,1))}
                    \end{figure}
              \item 2D Directivity Pattern at 2.44 GHz and $\phi = 0$
                    \begin{figure}[H]
                        \centering
                        \includegraphics[width=0.7\textwidth]{./image/figure3.png}
                        \caption{2D Directivity Pattern in dB}
                    \end{figure}
              \item 3D directivity pattern at 2.44 GHz
                    \begin{figure}[H]
                        \centering
                        \includegraphics[width=0.7\textwidth]{./image/figure4.png}
                        \caption{3D Directivity Pattern in dB}
                    \end{figure}
              \item Gain of the antenna at 2.44 GHz
                    \begin{figure}[H]
                        \centering
                        \includegraphics[width=0.7\textwidth]{./image/figure5.png}
                        \caption{Antenna Gain in dB}
                    \end{figure}

                    Since the gain is $0.44 \si{dB}$ and the directivity is $6.25 \si{dB}$, the antenna's radiation efficiency is $\frac{G_0}{D_0} = \frac{0.44}{6.25} = \boxed{7 \%}$.
          \end{enumerate}

\end{enumerate}




\end{document}
