\documentclass{article} % Defines the document class, article is commonly used
\usepackage[shortlabels]{enumitem}
\usepackage{amsmath}    % Allows for more advanced math formatting
\usepackage{amssymb}    % Provides additional mathematical symbols
\usepackage{amsthm}     % \qed
\usepackage{graphicx}   % image
\usepackage{float}      % image placement
\usepackage{hyperref}
\hypersetup{
    colorlinks=true,       % false: boxed links; true: colored links
    linkcolor=black,       % color of internal links
}
\usepackage[margin=0.5in]{geometry}

\begin{document}

\title{EEC133 Midterm Formula Sheet}
\author{Tao Wang}
\date{\today}

\maketitle

\begin{figure}[H]
    \centering
    \includegraphics[width=0.5\textwidth]{./image/figure1.png}
    \caption{S Parameter}
\end{figure}

\textbf{db to dbm}
\[P(\text{dBm}) = 10 \times \log(P(w)) + 30\]

\textbf{Steradian:}
\[A = r^2 \Omega\]

\begin{center}
    where A is the surface area patch, $r$ is the radius, and $\Omega$ is the solid angle.
\end{center}


\textbf{Half Power Beamwidth:}
\[\theta_2 - \theta_1\]

\begin{center}
    where $\theta_2, \theta_1$ are the angles where the normalized radiation intensity is $\frac{1}{2}$.
\end{center}

\textbf{Directivity:}

\[D_0 = \frac{4 \pi}{\Omega_p} = \frac{4 \pi}{\int_{sphere} F(\theta, \phi) d\Omega}\]

\begin{center}
    where $\Omega_p$ is the pattern solid angle.
\end{center}

\[\frac{A_e}{D_0} = \frac{\lambda^2}{4 \pi}\]

\begin{center}
    where $A_e$ is the effective area of the antenna, $D_0$ is the directivity of the antenna, and $\lambda$ is the wavelength of the wave transmitted by the antenna.
\end{center}

\textbf{Gain}
\[G_0 = e \times D_0\]

\begin{center}
    where $e$ is the efficiency of the antenna.
\end{center}

\textbf{Poynting Vector}
\[\vec{S}_{av} = \hat{z}\frac{|\widetilde{E}|^2}{2 \eta}\]

\textbf{Friss Transmission Equation:}
\[\frac{P_{rec}}{P_t} = e_t e_r \left(\frac{\lambda}{4\pi R}\right)^2 D_t D_r\]
\[=\left(\frac{\lambda}{4\pi R}\right)^2 G_t G_r\]
\[=\frac{e_r e_t A_t A_r}{\lambda^2 R^2}\]

\begin{center}
    where subscript t denotes the transmitter's parameter, r denotes the receiver's parameter, and R is the distance between two antennas.
\end{center}

\textbf{Vector Potential}
\[\vec{H} = \frac{1}{\mu_0} \nabla \times \vec{A}\]

\textbf{Retarded Potential}
\[\widetilde{A}(\vec{r}) = \int \frac{\mu_0 \widetilde{J}(\vec{r})e^{-jkr}}{4\pi R} dV'\]



\textbf{Hertzian Dipole Antenna:}
$l < \frac{\lambda}{50}$

Far Field ($r >> \lambda$)

\[\widetilde{E}_{ff, \theta} = j \eta_0 \frac{I k l}{4 \pi r} e^{-jkr} \sin(\theta)\]
\[\widetilde{H}_{ff, \phi} = \frac{\widetilde{E}_{ff, \theta}}{\eta_0}\]
\[D(\theta, \phi) = 1.5 \sin^2(\theta)\]
\[F(\theta, \phi) = \sin^2(\theta)\]
\[P_{rad} = 40 \pi^2 I_0^2\left(\frac{l}{\lambda}\right)^2\]
\[R_{rad} = 80 \pi^2 \left(\frac{l}{\lambda}\right)^2\]

\textbf{Small dipole Antenna}: $l < \frac{\lambda}{10}$

Fields are 1/2 those form the Hertzian Dipole.

\[P_{rad} = 10 \pi^2 I_0^2\left(\frac{l}{\lambda}\right)^2\]
\[R_{rad} = 20 \pi^2 \left(\frac{l}{\lambda}\right)^2\]

\textbf{Small Loop Antenna:}
$a << \frac{\lambda}{2 \pi}$

Far Fields ($r >> \lambda$)

\[\widetilde{H}_{\theta} = \frac{I k^2 (\pi a^2) }{4 \pi r} e^{-jkr}\sin(\theta)\]
\[\widetilde{E}_{\phi} = -\widetilde{H}_{\theta} \eta_0\]

\[D(\theta, \phi) = 1.5 \sin^2(\theta)\]
\[R_{rad} = 320 \pi^6 \left(\frac{a}{\lambda}\right)^4 N^2\]

Choosing Capacitor to be in parallel with loop antenna to cancel antenna's input reactance
\[j\omega C - j \frac{X_A}{R_A^2 + X_A^2} = 0\]

\begin{center}
    where $X_A$ is the input reactance and $R_A$ is the input resistance of the antenna
\end{center}

\end{document}
